\section{Axioms of Set Theory}

% Problem 1.1
\begin{problem}
  Verify $(a, b) = (c, d) \Leftrightarrow a = c and b = d$.
\end{problem}

\begin{solution}
  First we will show that $(a, b) = (c, d) \Rightarrow a = c \wedge b = d$.
  From definition, we have $\{\{a\}, \{a, b\}\} = \{\{c\}, \{c, d\}\}$.
  By extensionality we have $\{a\} = \{c\} \wedge \{a, b\} = \{c, d\}$ or $\{a\} = \{c, d\} \wedge \{a, b\} = \{c\}$.

  In the first case, by extensionality we have $a = c$ and $\{a, b\} = \{a, d\}$.
  Again, we have either $a = a \wedge b = d$ or $a = d \wedge b = a$.
  In both cases it follows that $a = c$ and $b = d$.

  If $\{a\} = \{c, d\}$ and $\{a, b\} = \{c\}$, by extensionality, we know that $a = c = d$ and $c = a = b$, hence again we have $a = c$ and $b = d$.

  The converse follows directly from extensionality.
\end{solution}

% Problem 1.2
\begin{problem}
  There is no set $X$ such that $P(X) \subset X$.
\end{problem}

\begin{solution}
  If we had $P(X) \subset X$ then for any $x \in P(X)$ it follows that $x \in X$.
  However since $X \in P(X)$ we would have $X \in X$, contradicting the axiom of regularity.
\end{solution}

% Problem 1.3
\begin{problem}
  If $X$ is inductive, then the set $\{x \in X : x \subset X\}$ is inductive.
  Hence $N$ is transitive, and for each $n$, $n = \{m \in N : m < n\}$.
\end{problem}

\begin{solution}
  Let $A = \{x \in X : x \subset X\}$.
  We wish to show that for every $x \in A$, $x \cup \{x\} \in A$.
  Since we know $x \in A$ it follows from the definition of $A$ that $x \subset X$ and $x \in X$.
  Hence, $x \cup \{x\} \subset X$ and thus $x \cup \{x\} \in A$.
  Trivially we see that the empty set is in $A$.
  Thus $A$ is inductive

  Since $N$ is inductive we can construct an inductive, transitive set $A$ such that $A \subset N$ by the above.
  However, by definition, $N \subset A$ since $A$ is an inductive set.
  Thus we have $N = A$ and $N$ must thus be transitive.
\end{solution}

% Problem 1.4
\begin{problem}
  If $X$ is inductive, then the set $\{x \in X : x \text{ is transitive}\}$ is inductive.
  Hence every $n \in N$ is transitive.
\end{problem}

\begin{solution}
  Note that the empty set is transitive, hence $\emptyset \in A$.
  For any $x \in A$ we want to show that $x \cup \{x\} \in A$.
  In other words for any transitive $x \in X$ we wish to show that $x \cup \{x\}$ is also transitive.
  If $t \in x \cup \{x\}$ then $t \in x$ or $t \in \{x\}$.
  If $t \in x$ we know $t \subset x \subset x \cup \{x\}$.
  If $t \in \{x\}$ then by extensionality we know $t = x$, hence $t \subset x \cup \{x\}$.
  Thus $x \cup \{x\}$ is transitive.

  Similarly to before, from $A \subset N$ and $N \subset A$ we have $N = A$, hence for any $n \in N$ we know that $n$ is transitive.
\end{solution}

% Problem 1.5
\begin{problem}
  If $X$ is inductive, then the set $\{x \in X : x \text{ is transitive and } x \notin x\}$ is inductive.
  Hence $n \notin n$ and $n \neq n + 1$ for each $n \in N$.
\end{problem}

\begin{solution}
  We can use the axiom of regularity to trivially get $x \notin x$ for any set $x$.

  If we don't wish to use the axiom of regularity we can do the following.
  Let $X$ be inductive and let $A$ be the subset of $X$ in the problem.
  Trivially $\emptyset \in A$.
  Let us now assume that $z \in A$, so $z$ is transitive and $z \notin z$.
  We will show that $z \cup \{z\}$ is also in $A$.
  Since $X$ is inductive and $z \in X$, we know that $z \cup \{z\} \in X$.
  By the same arguement as in problem 1.4, $z \cup \{z\}$ is transitive.
  We just need to show that $z \cup \{z\} \notin z \cup \{z\}$.
  Assume the opposite, let $z \cup \{z\} \notin z \cup \{z\}$.
  That means that $z \cup \{z\} \in z$ or $z \cup \{z\} \in \{z\}$.
  In the first case , by transitivity of $z$, we know that $z \cup \{z\} \subset z$, however clearly $z \subset z \cup \{z\}$, so we have $z = z \cup \{z\}$.
  This implies that $\{z\} \subset z$, so $z \in z$, contradicting our assumption that $z \in A$.
  On the other hand, if $z \cup \{z\} \in \{z\}$, then by extensionality $z \cup \{z\} = z$, which is, again, a contradiction.
  Thus $A$ is inductive.

  Again we can make the argument that $A = N$, so for every $n$ we have $n \notin n$ and $n \neq n + 1$, as that implies $n = n \cup \{n\}$ and $n \in n$.
\end{solution}

% Problem 1.6
\begin{problem}
  If $X$ is inductive, then $A = \{x \in X : x \text{ is transitive and every nonempty } z \subset x \text{ has an $\in$-minimal element}\}$ is inductive.
\end{problem}

\begin{solution}
  Again, it is easy to see $\emptyset \in A$ and for every $x \in A$ we know $x \cup \{x\}$ is transitive.

  Assume that $x \in A$ and that every nonempty $z \subset x$ has an $\in$-minimal element.
  We shall show that $x \cup \{x\}$ also has an $\in$-minimal element for every nonempty subset.
  Assume the contrary, there exists some $z \subset x \cup \{x\}$ such that $z$ has no $\in$-minimal element.
  It is clear that $x \in z$, otherwise $z \subset x$, contradicting our assumption.
  By definition of $z$, $x$ cannot be an $\in$-minimal element, so there exists an $s \in z$ such that $s \in x$.
  By transitivity of $x$, $s \subset x$, hence $s$ is $\in$-minimal in $x$ or $s = \emptyset$.

  If $s = \emptyset$, then $\emptyset \in z$, hence $z$ has an $\in$-minimal element, as $\emptyset$ is $\in$-minimal in every set.

  If $s$ is $\in$-minimal in $x$ then there is no $t \in x$ such that $t \in s$.
  We wish to show that $s$ is also $\in$-minimal in $x \cup \{x\}$
  Since we assumed that there is no $t \in x$ such that $t \in s$, all that is left to consider is $t = x$.
  If we had $t = x \in s$ then since $s \subset x$ we know that $x \in x$.
  Thus $\{x\} \subset x$, is a nonempty subset of $x$ that has no $\in$-minimal element in $x$, as $x \in \{x\}$ and $x \in x$.
  Hence, $s$ is $\in$-minimal in $x \cup \{x\}$ and in any subset of $x \cup \{x\}$.
\end{solution}

% Problem 1.7
\begin{problem}
  Every nonempty $X \subset N$ has an $\in$-minimal element.
\end{problem}

\begin{solution}
  Similar to before, $X = \{x \in X : x \text{ is transitive and every nonempty } z \subset x \text{ has an $\in$-minimal element}\} \subset N$ and, by exercise 1.6 we have $N \subset X$, hence $N = X$.
  So any nonempty subset of $N$ has an $\in$-minimal element.
\end{solution}

% Problem 1.9
\setcounter{problem}{8}
\begin{problem}[Induction]
  Let $A$ be a subset of $N$ such that $\emptyset \in A$, and if $n \in A$ then $n + 1 \in A$.
  Then $A = N$.
\end{problem}

\begin{solution}
  Clearly $A$ is inductive by it's definition.
  Thus we have $A \subset N$ and $N \subset A$.
  Hence $A = N$.
\end{solution}

% Problem 1.10
\begin{problem}
  Each $n \in N$ is T-finite.
\end{problem}

\begin{solution}
  We need to show that for each $n \in N$, every nonempty $X \subset P(n)$ has an $\subset$-maximal element.
  We shall show this by induction.
  Trivially, the assertion holds for $0$.
  Assume now that it holds for some $n \in N$.
  Consider a subset $X \in P(n + 1)$.
  If $n + 1 \in X$, then it is clear that $n + 1$ is a $\subset$-maximal element.

  Otherwise, if there is an $S \in X$ such that $n \in S$ pick the $S \in X$ with the maximal cardinality (need not be unique).
  We claim that this $S$ is $\subset$-maximal.
  If there was some $V \neq S$ such that $S \subset V$, this would contradict the fact that if $S \subset V$ then $|S| \le |V|$ with equality holding for finite sets only if $S = V$.

  If there is no $S \in X$ such that $n \in S$ then $X$ is a nonempty subset of $P(n)$, which by our assumption has an $\subset$-maximal element.

  Thus, by induction, every $n \in N$ is T-finite.
\end{solution}

% Problem 1.11
\begin{problem}
  $N$ is T-infinite; the set $N \subset P(N)$ has no $\subset$-maximal element.
\end{problem}

\begin{solution}
  Imagine there was some $n \in N$ that was $\subset$-maximal.
  Since $N$ is inductive, $n \cup \{n\} \in N$.
  However, $n \subset n \cup \{n\}$, a contradiction.
\end{solution}

% Problem 1.12
\begin{problem}
  Every finite set is T-finite.
\end{problem}

\begin{solution}
  Assume the opposite, let $S$ be a finite set that is T-infinite.
  That means there exists an $X \subset P(S)$ such that for every $u \in X$ there exists a $v \in X$ such that $v \neq u$.
  Notice that since $P(S)$ is finite, $X$ must also be finite.
  Let $X$ have $n$ elements.
  We can pick some $v_1 \in X$.
  There exists a $v_2 \in X$ such that $v_1 \neq v_2$ and $v_1 \subset v_2$.
  Similarly there must be some $v_3$ such that $v_2 \neq v_3$ and $v_2 \subset v_3$.
  Notice that $v_1 \subset v_3$ and $v_1 \neq v_3$, otherwise $v_1 \subset v_2 \subset v_3$ would imply $v_1 = v_2 = v_3$, which contradicts our choice of $v_i$.
  We can continue like this until we have $v_1 \subset v_2 \ldots \subset v_n$, where $v_1 \neq \dots \neq v_n$.
  Since $v_n \in X$, there is some $v_{n + 1}$ such that $v_n \subset v_{n + 1}$ and $v_n \neq v_{n + 1}$.
  By the above reasoning we know that $v_{n + 1} \neq v_i$ for any previous $i$.
  This would mean that $X$ has $n + 1$ distinct elements, which contradicts our assumption.
  The theorem follows from this contradiction.
\end{solution}

% Problem 1.13
\begin{problem}
  Every infinite set is T-infinite.
\end{problem}

\begin{solution}
  Let $A$ be an infinite set.
  Consider $X = \{u \in A : u \text{ is finite}\}$.
  It is clear that $X$ is a nonempty subset of $P(A)$.
  As it contains every singleton subset of $A$ it is clear that $X$ is infinite.
  We shall now show that no element of $X$ is $\subset$-maximal.
  Let $u \in X$ and let $\alpha \in A$ such that $\alpha \notin u$ (since $u$ is finite such an element must exist).
  Clearly $\{\alpha\} \in X$.
  Consider the set $u \cup \{\alpha\}$, we see that it belongs to $X$ as it is a union of two finite sets.
  It is also clear that $u \subset u \cup \{\alpha\}$, hence $u$ is not $\subset$-maximal.
  Since our choice of $u$ was arbitrary, no element of $X$ is $\subset$-maximal.
  Thus $A$ is T-infinite.
\end{solution}

% Problem 1.14
\begin{problem}
  The Separation Axioms follow from the Replacement Schema.
\end{problem}

\begin{solution}
  Let $X$ be a set and $\varphi(x, p)$ be a property.
  We wish to show that there exists a set $Y = \{u \in X : \varphi(x, p)\}$.
  We shall define a function $F = \{(x, x) : \varphi(x, p)\}$.
  Since $X$ is a set, by replacement $F(X) = \{F(x) : x \in X\}$ is a set.
  By the definition of our function $F(X)$ is precisely $F(x) = x$ if and only if $\varphi(x, p)$, so clearly $F(X) = \{x \in X : \varphi(x, p)\}$ is a set.
\end{solution}

% Problem 1.15
\begin{problem}
  Instead of Union, Power Set, and Replacement Axioms consider the following weaker versions:
  \begin{enumerate}
    \item $\forall X \exists Y \bigcup X \subset Y$     i.e., $\forall X \exists Y (\forall x \in X) (\forall u \in x) u \in Y$;
    \item $\forall X \exists Y P(X) \subset Y$;     i.e., $\forall X \exists Y \forall u (u \subset X \Rightarrow u \in Y)$;
    \item If a class $F$ is a function, then $\forall X \exists Y F(X) \subset Y$.
  \end{enumerate}
  Then axioms 1.4., 1.5., and 1.7. can be proved from (1.), (2.), and (3.), using the Separation Schema.
\end{problem}

\begin{solution}
  We simply specify the subset of $Y$ equal to our desired axiom.

  For example let us assume $\forall X \exists Y (\forall x \in X) (\forall u \in x) u \in Y$.
  Let $\bigcup X = \{u \in Y : \exists z (z \in X \wedge u \in z)\}$, hence we have the union axiom.
  The other axioms follow mutatis mutandis.
\end{solution}
