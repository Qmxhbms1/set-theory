\section{Cardinal Numbers}

% Problem 3.5
\setcounter{problem}{4}
\begin{problem}
  Show that $\Gamma (\alpha \times \alpha) \le \omega^{\alpha}$.
\end{problem}

\begin{solution}
  We will use induction.
  The claim trivially holds for $0$.
  If the theorem holds for $\alpha$, we will consider $\alpha + 1$.
  We have $\Gamma(\alpha + 1 \times \alpha + 1) = \Gamma(\alpha \times \alpha) + \alpha \cdot 2 + 1$, as we know that the amount of pairs with $max(\xi, \eta) = \alpha$ is $\alpha + \alpha + 1$.
  By the inductive hypothesis and using the easily proven fact that $1 \le \alpha \le \omega^{\alpha}$ we get
  \[\Gamma(\alpha + 1 \times \alpha + 1) \le \omega^{\alpha} + \omega^{\alpha} + \omega^{\alpha} + \omega^{\alpha} = \omega^{\alpha} \cdot 4 < \omega^{\alpha} \cdot \omega = \omega^{\alpha + 1}.\]
  Thus we have proven the induction on the successor ordinals.
  If $\alpha$ is a limit ordinal and we assume the statement holds for all preceding ordinals then by continuity of $\Gamma$ we have
  \[\Gamma(\alpha \times \alpha) = \bigcup \Gamma(\beta \times \beta) \le \bigcup \omega^{\beta} = \omega^{\alpha}.\]
  This concludes the proof.
\end{solution}

% Problem 3.6
\begin{problem}
  There is a well-ordering of the class of all finite sequences of ordinals such that for each $\alpha$, the set of all finite sequences in $\omega_{\alpha}$ is an initial segment and its order-type is $\omega_{\alpha}$.
\end{problem}

\begin{solution}
  Consider the following well-ordering:

  Let $\{\alpha_i\}$ and $\{\beta_j\}$ be two finite sequences with $i = 0, 1, \ldots, n$ and $j = 0, 1, \ldots, m$, for some finite numbers $n, m$.
  First we define the function $f : Ord^{<\omega} \to Ord$ by $f(s_i) = \sum s_i + n$, where $n$ is the length of the sequence $s_i$.
  We will say that $\{\alpha_i\} < \{\beta_i\}$ if:
  \begin{enumerate}[label=(\alph*)]
    \item $f(\alpha_i) < f(\beta_i)$;
    \item $f(\alpha_i) = f(\beta_i)$, and $\alpha_i$ preceds $\beta_i$ in the lexicographical ordering.
  \end{enumerate}

  The fact that this is a well-ordering follows from $f$ being a function defined on $Ord$, which is well-ordered and the fact that the lexicographical ordering is a well-ordering.
  If we had a sequence $s$ which contains an ordinal greater than $\omega_{\alpha}$, then we would have $f(s) = \sum s_i + n > \omega_{\alpha} + n$.
  On the other hand, any sequence $t$ in $\omega_{\alpha}^{<\omega}$ must have $f(t) < \omega_{\alpha}$, because $\omega_{\alpha}$ is a limit ordinal and thus it cannot be reached with a finite sum.
  Since the cardinality of the set $\omega_{\alpha}^{<\omega}$ is $\aleph_{\alpha}$, it follows that the order-type of any ordering must be at least $\omega_{\alpha}$.
  It is quite clear that any initial segment $s$ of $\omega_{\alpha}^{<\omega}$ must have an order-type of less than $\omega_{\alpha}$ because $|s| < |\omega_{\alpha}^{<\omega}| = \aleph_{\alpha}$.
  Using the fact that the order-type of a well-ordered set $\{K, <\} = \omega_{\alpha}$ if $|K| = \aleph_{\alpha}$ and every initial segment of $K$ has an order-type of less than $\omega_{\alpha}$ we get that the order-type of $\omega_{\alpha}^{<\omega}$ is $\omega_{\alpha}$ and we are done.
\end{solution}

% Problem 3.7
\begin{problem}
  If $B$ is a projection of $\omega_{\alpha}$, then $|B| \le \aleph_{\alpha}$.
\end{problem}

\begin{solution}
  Let $f$ be a mapping of $\omega_{\alpha}$ onto $B$.
  Let us consider the quotient set of $\omega_{\alpha}$ consisting of equivalence classes defined by $f(x) = f(y)$.
  Since the equivalence classes are sets of ordinals, it follows that there is a least ordinal for each equivalence class, pick it as the representative.
  Now we have a set $K \subset \omega_{\alpha}$ of all representatives of the equivalence classes.
  It is obvious that the restriction of $f$ to $K$ is a bijection by construction.
  Thus we have $|K| = |B|$.
  Since $K \subset \omega_{\alpha}$ it follows that $|B| = |K| \le \aleph_{\alpha}$.
\end{solution}

% Problem 3.8
\begin{problem}
  The set of all finite subsets of $\omega_{\alpha}$ has cardinality $\aleph_{\alpha}$.
\end{problem}

\begin{solution}
  Since the set of all finite subsets contains all singleton sets, there is a very natural injection from $\omega_{\alpha}$ to $\omega_{\alpha}^{<\omega}$ defined by $f(x) = \{x\}$.
  Thus $|\omega_{\alpha}^{<\omega}| \ge \aleph_{\alpha}$.
  We can write $|\omega_{\alpha}^{<\omega}| = |P_1 (\omega_{\alpha})| + |P_2 \omega_{\alpha}| + \ldots$, since these are all disjoint sets.
  All of these sets will have cardinality $\aleph_{\alpha}$, as we can easily define an injection from $P_n (\omega_{\alpha})$ to $\aleph_{\alpha}^{n}$, simply let $f(S) = (x_1, x_2, \ldots)$, where $S = \{x_1, \ldots\}$ and we order it using the well ordering of the ordinals so that $x_1 < x_2 < \ldots$.
  Since we know that $\aleph_{\alpha} \cdot \aleph_{\alpha} = \aleph_{\alpha}$ we end up with
  \[\omega_{\alpha}^{<\omega}| = |\aleph_{\alpha}| \cdot \omega.\]
  Finally, since $|\omega| = \aleph_0$ and $\aleph_{\alpha} \cdot \aleph_{\beta} = \max (\aleph_{\alpha}, \aleph_{\beta})$ we get
  \[|\omega_{\alpha}^{<\omega}| = \aleph_{\alpha}.\]
\end{solution}

% Problem 3.9
\begin{problem}
  If $B$ is a projection of $A$, then $|P(B)| \le |P(A)|$.
\end{problem}

\begin{solution}
  Let $f$ be a function from $A$ onto $B$.
  Consider the function $g: P(B) \to P(A)$ defined as $g(X) = f^{-1} (X)$.
  This function is well-defined because every subset of $B$ has a unique inverse image.
  Let $X, Y \in P(B)$ be such that $g(X) = g(Y)$.
  This trivially follows from the fact that $f$ is a surjective function.
  For any $x \in X$ there exists some $a \in A$ such that $f(a) = x$.
  The set of all such $a \in A$ forms the set $S = g(X)$.
  However, it also forms the set $g(Y)$, meaning that both $X$ and $Y$ are images of the set $S$.
  Since the image of a function is unique it follows that $X = Y$ and thus $g$ is injective.
  From there $|P(B)| \le |P(A)|$.
\end{solution}

% Problem 3.10
\begin{problem}
  $\omega_{\alpha + 1}$ is a projection of $P(\omega_{\alpha})$.
\end{problem}

\begin{solution}
  From Theorem 3.5 we know that there exists a bijection from $\omega_{\alpha} \times \omega_{\alpha}$ to $\omega_{\alpha}$.
  This implies that there is a bijection from $P(\omega_{\alpha} \times \omega_{\alpha})$ to $P(\omega_{\alpha})$.
  Thus if we find that $\omega_{\alpha + 1}$ is a projection of $P(\omega_{\alpha} \times \omega_{\alpha})$, by composition it follows that it is also a projection of $P(\omega_{\alpha})$.
  Let $R \subset \omega_{\alpha} \times \omega_{\alpha}$ be a well-ordering and let $f(R)$ be its order-type.
  We claim that this $f$ is a surjective mapping onto $\omega_{\alpha + 1}$.
  It is trivial to see that for any $\xi < \omega_{\alpha}$ we can find an $X$ such that $f(X) = \xi$ by just considering the canonical well-ordering of the ordinals.
  % Not finished, come back to at some point
\end{solution}

% Problem 3.11 requires the previous problem

% Problem 3.12
\setcounter{problem}{11}
\begin{problem}
  If $\aleph_{\alpha}$ is an uncountable limit cardinal, then $cf \omega_{\alpha} = cf \alpha$; $\omega_{\alpha}$ is the limit of a cofinal sequence $\langle \omega_{\xi} : \xi < cf \alpha \rangle$ of cardinals.
\end{problem}

\begin{solution}
  By the definition of cofinality, there exists some increasing, cofinal sequence $\langle \alpha_{\eta} : \eta < cf \alpha \rangle$.
  We consider the sequence $\langle \omega_{\alpha_{\eta}} : \eta < cf \alpha \rangle$.
  Clearly this sequence will also be increasing by the definition of the alephs.
  By the choice of our sequence, the limit will be $\omega_{\alpha}$ and we have shown that $cf \omega_{\alpha} \ge cf \alpha$.

  To show the converse, take an increasing, cofinal sequence $\langle \omega_{\xi} : \xi < cf \omega_{\alpha} \rangle$.
  We consider the cardinality of each of these $\omega_{\xi}$.
  By the definition of cardinals, it follows that $\aleph_{xi} < \aleph_{\alpha}$, so we can write $\aleph_{\xi} = \aleph_{\beta_{\eta}}$ for some ordinal $\beta_{\eta} < \alpha$.
  We claim that the sequence $\langle \beta_{\eta} : \eta < cf \omega_{\alpha} \rangle$ is cofinal in $\alpha$.
  Trivially this is an increasing sequence since it stems from the increasing sequence of $\omega_{xi}$.
  We've also established already that $\alpha$ is an upper bound of this sequence.
  If there was some $\gamma < \alpha$ which was also an upper bound then the supremum of $\omega_{\beta{\eta}} : \eta < cf \omega_{\alpha}$ would be $\omega_{\gamma}$, contradicting that this sequences is cofinal in $\omega_{\alpha}$.
  From there, the theorem follows.
  % I'm fairly certain I just made up half the second part out of thin air and that non of it actually holds
\end{solution}
